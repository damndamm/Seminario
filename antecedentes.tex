\section{Antecedentes}
	
	En el pasado, la tecnología VoIP era de baja calidad, puesto que los algoritmos 
	de compresión no eran sofisticados y la capacidad de transmisión de los medios 
	digitales era bastante baja, sin embargo, la tecnología ha avanzado lo suficiente 
	como para incluso mostrarse como un competidor robusto contra la telefonía tradicional.
	 
	Aunque la tecnología VoIP no es precisamente reciente, el auge actual en las tecnologías 
	de comunicación, así como el aumento de la velocidad de los enlaces digitales y el alto 
	costo de las tarifas telefónicas han proyectado el crecimiento de la tecnología VoIP en los últimos años.
	
	VoIP inicia casi inmediatamente después de la liberación de Internet (1994) para uso público.
	En 1995 un grupo de jóvenes israelíes que pretendían transmitir voz de un computador a 
	otro crearon el primer “softphone” (programa de computador que emula el comportamiento 
	de un teléfono) llamado Internet Phone Software o Vocaltech phone, desafortunadamente 
	la tecnología en aquel momento limitaba mucho la calidad del servicio, por lo que no 
	tuvo mucho éxito. Luego, en 1997 aparecen las primeras PBX (Private Branch Exchange), 
	prolifera el uso de protocolo H323 para transmisión de audio y video, aunque a baja 
	calidad por el ancho de banda costoso y de baja capacidad. Para el año 2000 ya Asterisk 
	y el protocolo SIP habían ingresado en el mercado VoIP, aunque las empresas todavía no 
	confían en Linux por lo que se mantienen usando PBX privativas y H323. 
	
	En el año 2003 Skype entra en el mercado, Asterisk introduce el protocolo IAX y el 
	costo de los teléfonos IP se reduce un 50\%. En los próximos años surge la Astricon, 
	una convención internacional de usuarios de Asterisk, se desarrollan mejoras en Skype 
	y en el protocolo IAX (surge IAX2), se reducen más los precios de los terminales VoIP, 
	Google incursiona en la tecnología al lanzar GoogleTalk. Para 2006 Skype logra alcanzar 
	50 millones de usuarios. En los siguientes años surgieron otros servicios como Google 
	Voice en 2009 y Viber en 2010. La era móvil ha impulsado la masificación de VoIP al 
	aprovechar una de sus principales características, la movilidad, para el 2012 las 
	aplicaciones móviles como Line, Yuliop y Whatsapp incluyen llamadas gratuitas entre 
	usuarios a través de internet. 
	
	En los últimos años la tecnología VoIP ha tomado gran auge gracias a las mejoras 
	en la infraestructura de la red y al interés por reducir cada vez más los costos 
	asociados a las telecomunicaciones, principalmente en empresas pequeñas o 
	emprendimientos.