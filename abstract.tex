\documentclass[12pt]{article}
\usepackage[spanish]{babel}
\title{Lineamientos para Soluciones de Alta Disponibilidad en VoIP}
\author{Br. Sebastian Suarez \and
		Br.Dorjes Molina}
\date{Noviembre 2016}

\begin{document}
	\pagenumbering{gobble}
	\maketitle
	\newpage
	\begin{abstract}
		\item
		La tecnolog\'ia usada tradicionalmente para la transmisi\'on de conversaciones vocales, se basa en el concepto de conmutaci\'on de circuitos, es decir, en el establecimiento de un circuito f\'isico que permanece reservado durante el tiempo de la llamada, esto se traduce en que los recursos usados en esta llamada no pueden ser utilizados por otra llamada hasta que la primera termine, incluso durante las pausas o silencios que existan en la conversaci\'on.
		
		Por otro lado, la tecnolog\'ia usada para transportar datos se basa en el concepto de conmutaci\'on de paquetes, es decir, una misma comunicaci\'on sigue diferentes caminos entre origen y destino mientras dura, lo que se traducen en que los recursos que intervienen en una conexi\'on pueden ser utilizados por otras conexiones que se efect\'uen al mismo tiempo.
		
	\end{abstract}

\end{document}