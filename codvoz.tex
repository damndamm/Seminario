	
\section{Codificación de Voz}

Las redes de datos transfieren la información de modo digital 
y en VoIP la información que se maneja son señales de audio, 
principalmente señales de voz. Estas señales son analógicas 
por naturaleza, así que se requiere de un mecanismo que 
digitalice las señales analógicas, es decir convierta las 
señales de voz en una secuencia de números discreta, para 
luego poder ser enviadas por la red. Sí la solución VoIP 
implementada es completamente digital, este proceso ocurre 
directamente en el teléfono (teléfonos digitales, celulares, 
micrófonos de computadores); en cambio si el sistema incluye 
secciones combinadas con telefonía tradicional, el proceso de 
digitalización ocurre en las intersecciones de los sistemas 
bien sea en las pasarelas (gateways) del sistema o en conectores 
ATA (Analog Telephone Adapter) como se aprecia en las siguientes 
figuras [FIGURA \ref{fig:pasarela}]  [FIGURA \ref{fig:ata}].

	\begin{figure}[h]
		
		%nombre de la imagen, sin extencion. "width=\textwidth" ancho igual al texto
		\includegraphics[width=\textwidth]{pasarela}
		
		%titulo de la imagen, salen debajo de la imagen y en el indice de imagenes.
		\caption{VoIP combinada con telefonía tradicional a través de una pasarela}
		
		%centrado, por si las moscas
		\centering
		
		%para referencias
		\label{fig:pasarela}
	\end{figure}

Los primeros trabajos sobre digitalización de audio fueron realizados 
por el Ingeniero Alec Reeves poco antes de la segunda guerra mundial, quien desarrolló un sistema de audio digital con fines militares, 
sin embargo, la tecnología de comunicaciones de la época todavía 
no estaba lista para dicho avance. Alec Reeves patento un total 
de 82\footnote{ http://www.quantium.plus.com/ahr/patents.htm } 
inventos entre ellos se destaca la idea de la Modulación por 
impulsos codificados (PCM, Pulse Coded Modulation). Al rededor 
de los 60s fue que se popularizo la tecnología de PCM 
pero para entonces ya no eran reclamables los derechos de la 
patente. 

	
		\begin{figure}[h]
			
			%nombre de la imagen, sin extencion. "width=\textwidth" ancho igual al texto
			\includegraphics[width=\textwidth]{ata}
			
			%titulo de la imagen, salen debajo de la imagen y en el indice de imagenes.
			\caption{VoIP usando teléfonos analógicos y ATAs}
			
			%centrado, por si las moscas
			\centering
			
			%para referencias
			\label{fig:ata}
		\end{figure}
		
		
Inicialmente la digitalización de voz se basó en codificar la forma 
de onda de la señal analógica mediante un proceso de muestreo, 
cuantificación y codificación de la señal, el proceso de PCM. Luego, con el objetivo de reducir la tasa de bits requeridas para transmitir la señal, se introdujeron las técnicas predictivas y comenzaron codificar solo la diferencia entre los valores de las muestras reales y la predicción de la señal en base a la extrapolación de las muestras anteriores. 

\subsection{Códecs}

Los códecs son algoritmos o dispositivos que realizan la codificación y 
decodificación de la señal de audio, son usados a menudo en videoconferencias y emisiones de medios de comunicación. La 
mayoría de los códecs provoca perdidas de información ya que 
están diseñados para minimizar el tamaño de los datos digitales. 
Existen códecs sin pérdida (lossless) que mantienen la calidad 
más precisa del sonido pero generan datos digitales muy grandes 
que en VoIP no son necesarios. 

Existe gran variedad de códecs en el mercado actual y su clasificación puede depender de varios aspectos como la técnica que usan para codificar, el ancho de banda que se muestrea o incluso la tasa de bits resultante. En la [FIGURA \ref{fig:tablacodecs}] se puede apreciar las principales diferencias entre los codecs más comunes.


\subsection{Codificación de Video}

	