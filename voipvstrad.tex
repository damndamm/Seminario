	
\section{VoIP frente a la telefonía tradicional}
	
	Las capacidades de la telefonía tradicional están directamente relacionadas con la cantidad de conexiones físicas de la red, ya que cada llamada debe establecerse a través de un circuito. Por tanto, muchas de las limitaciones de la telefonía tradicional surgen principalmente de la tecnología de conmutación de circuitos 
	usada por esta.
	
	En telefonía tradicional, las compañías tardan más tiempo en implementar nuevas características (como llamadas en espera o llamadas en 3 vías) e incluso pueden no ser implementadas en toda la red al mismo tiempo. La fidelidad del sonido está limitada al ancho de banda disponible entre el emisor y el destino de la 
	llamada, y el número máximo de llamadas entre dos locaciones está limitado a la disponibilidad de circuitos de voz disponibles entre ellas. Aumentar la capacidad del sistema implica aumentar la capacidad de circuitos y esto se traduce en una inversión monetaria. 
	 
	Las compañías de telefonía tradicional han logrado hacer muchos avances en identificar y resolver problemas relacionados con capacidad y costo. La implementación de líneas T1 o T3 han reducido los costos de la telefonía, la introducción de características como rutas de menor costo (LCR, least cost routing) a las PBX han reducido los costos de las llamadas de larga distancia.
	 
	En su momento, las características de la telefonía tradicional fueron consideradas como ventajas competitivas. Estas tecnologías, al ser  implementadas por una empresa, se volvían parte del costo de “hacer negocios”. Pero, la llegada de la Internet trajo consigo innovadores en busca de nuevas soluciones y las diferencias en la ingeniería y filosofía de estas tecnologías parecían indicar que el Internet era superior a las redes de voz tradicionales en muchos sentidos.
	 
	En la Internet, los protocolos de comunicación son mejorados constantemente por lo que nuevas características pueden ser introducidas rápidamente mientras el ancho de banda mejora y los costos de la red se reducen. En la Internet, o las redes basadas en tecnología IP en general, la capacidad está más directamente relacionada a la eficiencia del software y no tanto a la 
	capacidad física como la telefonía tradicional; por tanto, las redes IP crecen a medida que el software mejora mientras las redes telefónicas requieren hardware adicional para agregar capacidad. 
	 
	Un punto a favor de la telefonía tradicional es su fuente de alimentación, al proveer energía para el funcionamiento del sistema desde la oficina central, por lo que en un corte de corriente los teléfonos tradicionales pudieran ser los únicos operativos. En cambio, en un entorno VoIP, las fuentes de energía 
	de respaldo pueden mantener el sistema funcionando por un tiempo limitado y  tecnologías como PoE (Power over Ethernet, Corriente a través de ethernet) tienen corto alcance por lo que un corte de corriente también deshabilita el sistema de comunicaciones. Aunque la telefonía tradicional es confiable, no es a prueba de desastres, la caída de un enlace o ruptura de un circuito 
	puede dejar incomunicada grandes zonas hasta que sea reparado el problema. Las redes IP permiten redundancia y mecanismos de recuperación de bajo costo y fáciles de implementar. Evitar interrupciones locales en la conexión es más fácil de lograr con redes IP que con telefonía. 
	  
	VoIP es vagamente definida como usar el paquete de protocolos TCP/IP para facilitar conversaciones de voz, pero es mucho más que eso. VoIP puede ser usada para reemplazar la telefonía tradicional empresarial y doméstica, o incluso solo agregar características a sistemas de telefonía tradicional. 
	
	VoIP permite conectar PBX tradicionales en lugares remotos, facilitar comunicaciones de voz entre distintas aplicaciones, facilita la transferencia de video en tiempo real, conferencias hasta mensajería instantánea. Sin embargo, sigue siendo una 
	tecnología joven que debe resolver sus problemas de estabilidad, seguridad y calidad de servicio antes que sus detractores comiencen a aceptar sus beneficios.