%bibliografia, el número indica la cantidad de entradas que tendrá, no puede ser mayor a 99
	\begin{thebibliography}{15}
		\bibitem{switching} 
		T. Eallingford, Switching to VoIP, Beijing: O'Reilly, 2005. .
		
		\bibitem{voiptoip} 
		R. Quispe y G. Suárez.	
		\textit{ Voz sobre IP y Telefonía sobre IP}.
		
		\bibitem{nyquist} 
		H. Nyquist, Certain topics in telegraph transmission theory, de Trans. AIEE, Volumen 47, 1928.
		
		\bibitem{packet}
		B. Hartpence, Packet Guide to Voice over IP, Gravenstein, Sebastopol: O'Reilly Media, Inc, 2013. 
		
		\bibitem{understanding}
		N. Wittenberg, Undertanding Voice over IP Technology, Clifon Park, NY: Delmar Cengage Learning, 2009.
		
		\bibitem{rtp}
		C. Perkings, RTP: Audio and Video for the Internet, Boston: Addison-Wesley, 2003.
		
		\bibitem{undersip}
		A. Jhonston, SIP: Understanding the Session Initiation Protocol, Norwood, Massachusetts: Artech House, 2009.
		
		\bibitem{rfcsip}
		J. Rosenberg, H. Schulzrinne, G. Camarillo, A. Johnston, J. Peterson, R. Sparks, M. Handley y E. Schooler, RFC 3261 SIP: Session Initiation Protocol, 2002. 
		
		\bibitem{siphandbook}
		S. Ilvas y M. Ahson, SIP Handbook, Boca Raton, Florida: CRC Press, 2009.
		
		\bibitem{codif}
		J. Joskowicz, Codificación de Voz y Video, Montevideo, Uruguay: Universidad de la Republica, 2015.
		
		\bibitem{g711}
		Recommendation G.711: "Pulse Code Modulation (PCM) of voice frequencies", CCITT, 1988.
		
		\bibitem{cartagena}
		A. Rodríguez, Instalación de un sistema VoIP corporativo basado en Asterisk, Cartagena: Universidad Politecnica de Cartagena, 2008. 
		
	\end{thebibliography}