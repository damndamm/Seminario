\section{Características}

	La telefonía IP puede realizar las mismas funciones de la telefonía tradicional 
	e incluso ofrece funciones adicionales como: Monitoreo de llamadas, Recuperación 
	de llamadas, Transferencia de llamadas, Grabación de llamadas, Identificación de 
	usuarios, Videoconferencias, Mensajería SMS, Música en espera, Llamadas en espera, 
	Recepción y transmisión de fax, Llamadas de emergencia, Llamadas automáticas, 
	entre otras.
	
	Las ventajas de la telefonía IP sobre la telefonía tradicional no se limitan 
	solo a funciones adicionales, sino que también presentan un conjunto de 
	bonificaciones entre las que se destaca: 
	
	\begin{itemize}
		\item Reducción de costos de instalación: VoIP requiere de una infraestructura 
		similar a la red de datos, por tanto, se puede integrar en la misma red o 
		aprovechar parte de la red existente. 
		\item Reducción de costos de mantenimiento: Como el equipo de VoIP es similar 
		al de red de computadoras ya instalada, el equipo técnico ya está familiarizado 
		con problemas habituales de conectividad, por lo que no se requiere de contratar 
		a terceros para solucionar problemas.
		\item Escalabilidad: Aumentar la capacidad de la red es micho más sencillo 
		comparado con telefonía tradicional y las inversiones monetarias son incluso
		 menores.
		\item Seguridad: Aunque la seguridad  en VoIP depende principalmente de la 
		seguridad de la red de datos, VoIP posee funcionalidades adicionales como 
		cifrado de la voz digitalizada o autenticación, autorización y protección 
		de los datos que viajan en su red.
		\item Compatibilidad:  Los estándares de la industria facilitan la compatibilidad
		 entre dispositivos o programas de distintos fabricantes, lo que facilita el 
		 acceso a la tecnología.
		\item Integración de servicios: Si bien VoIP sirve como plataforma para una central 
		telefónica, la gran variedad de funcionalidades adicionales que posee aumentan su 
		potencial. Integrar voz, video y datos en una misma red es uno de los pilares
		 de VoIP.
		\item Calidad de servicio: Permite asignar prioridad a los datagramas que viajan 
		por el medio lo que permite garantizar la transmisión de la conversación bajo un 
		umbral de considerable calidad.
		\item Movilidad: Siempre que se tenga una conexión a Internet, su servicio de 
		VoIP estará disponible en cualquier parte del mundo, podrá enviar y recibir 
		llamadas desde el mismo número de teléfono o extensión. 
	\end{itemize}